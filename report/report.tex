\documentclass{article}

\usepackage[portuguese]{babel}
\usepackage[utf8]{inputenc}
\usepackage{minted}
\usepackage{titling}
\usepackage{mdframed}
\usepackage{indentfirst}

\definecolor{codebgcolor}{rgb}{0.98, 0.98, 0.98}

\newcommand{\subtitle}[1]{%
  \posttitle{%
    \par\end{center}
    \begin{center}\large#1\end{center}
    \vskip0.5em}%
}

\title{Técnicas de Programação Avançada}
\subtitle{Trabalho 1 -- Aclimatação com Juízes Online de Programação}
\date{\today}
\author{Matheus da Silva Garcias}

\begin{document}
    \pagenumbering{gobble}
    \maketitle
    \newpage
    \pagenumbering{arabic}

    \section{Introdução}
        \subsection{Introdução ao documento}
            Este documento é rum relatório de problemas resolvidos na plataforma Kattis. Através desse documento será apresentada uma breve introdução, logo após serão apresentados os códigos e comentários sobre suas soluções.
        \subsection{O que é a plataforma Kattis}
            A plataforma Kattis é um website que permite que estudantes enviem soluções de problemas de computação variados, em nível de dificuldade e assunto, propostos pelo mesmo.
        \subsection{Quem corrige os códigos enviados}
            A verificação se dá a partir da entrada e saída de dados do programa. Cada problema define um modelo de entrada de dados e um modelo de saída. O programador deve interpretar a saída e entrada de dados de acordo com os formatos definidos pelo problema.
            \par Após o envio do programa para o juiz virtual, a é executa uma verificação baseado no saída de dados do seu programa com os resultados reais. Caso todos as saídas do seu programam sejam iguais ao resultados dos juiz, seu programa é aprovado.

    \section{Problemas}
        \subsection{A New Alphabet}
            A resolução deste problema é simples, basta trocar cada caractere da string fornecida pelos caracteres equivalentes fornecidos pela tabela do problema.

\begin{mdframed}[linewidth=0pt,backgroundcolor=codebgcolor]
    \inputminted[breaklines]{python}{../src/anewalphabet/anewalphabet.py}
\end{mdframed}

\par Basicamente todos os caracteres lidos são transformados para minúscula, casa haja algum maiúsculo.
Uma tabela para conversão dos caracteres é pré-criada para uso posterior no código.
\par Na parte do código a seguir, como apenas caracteres alfabéticos serão convertidos, é vericado se são caracteres a partir de seus números na tabela ASCII, já que são sequenciais. Caso seja alfabético, é subtraído 97 de seu valor ASCII, porque 97 é o valor de "a", a primeira letra do alfabeto. Assim podemos usar essa subtração para obter o indíce da lista de símbolos.

\begin{mdframed}[linewidth=0pt,backgroundcolor=codebgcolor]
    \begin{minted}[]{python}
        for l in t:
            if ord(l) >= 97 and ord(l) <= 122:
                print(symbols[ord(l) - 97])
            else:
                print(l)
    \end{minted}
\end{mdframed}
\par Caso o caractere não esteja dentro dos valores da tabela ASCII de caracteres alfabéticos minúsculos, então ele é enviado para a saída padrão sem nenhum processamento.

    
\end{document}